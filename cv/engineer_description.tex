\documentclass[11p]{article}
\usepackage{times}

% *** PAGE SETTING ***
\oddsidemargin .25in  %  Note \oddsidemargin = \evensidemargin
\evensidemargin .25in \marginparwidth 0.07 true in
%\marginparwidth 0.75 true in
%\topmargin 0 true pt      % Nominal distance from top of page to top of
%\topmargin 0.125in
\topmargin -0.5in \addtolength{\headsep}{0.25in}
\textheight 8.5 true in    % Feight of texst (including footnotes & figures)
\textwidth 6.0 true in    % Width of text line.
\widowpenalty=10000 \clubpenalty=10000

\parindent 0pt
\topsep 4pt plus 1pt minus 2pt
\partopsep 1pt plus 0.5pt minus 0.5pt
\itemsep 2pt plus 1pt minus 0.5pt
\parsep 2pt plus 1pt minus 0.5pt
\parskip .5pc

% *** MATH PACKAGES ***
\usepackage{amsmath}
\usepackage{amssymb}
\usepackage{bm}
\usepackage{amsthm}

% *** ALGORITHM PACKAGES ***
\usepackage{algorithm}
\usepackage{algpseudocode}
\algrenewcommand\alglinenumber[1]{\normalfont #1.}
\algnewcommand{\LeftComment}[1]{\Statex \(\triangleright\) #1}

% *** GRAPHICS PACKAGES ***
\usepackage{graphicx}
\usepackage{pgfplots}
\usepackage{float}
\usepackage{caption}
\usepackage{subcaption}
\usepackage{pdflscape}

\usepackage{url}

% *** USER DEFINITION ***
\newtheoremstyle{problemstyle} % <name>
{3pt}                  % <space above>
{3pt}                  % <space below>
{\normalfont}          % <body font>
{}                     % <indent amount}
{\bfseries}            % <theorem head font>
{\normalfont:}         % <punctuation after theorem head>
{.5em}                 % <space after theorem head>
{}                     % <theorem head spec (can be left empty, meaning `normal')>
\theoremstyle{problemstyle}

\title{Description of the Control and Automation Engineer Degree (Talented Engineer's Program)\\issued by Hanoi University of Science and Technology}

\author{}

\date{}

\begin{document}

\maketitle

\section{General Information}

Institution: Hanoi University of Science and Technology

Department: 

Major: Control and Automation Engineering

Specialty: Automatic Control

Duration: 5 years

Mode of study: Full-time

Language of training and exams: Vietnamese

\subsection{On the talented engineer's program}

Selected through competitive exams on maths and physics after the university entrance exam.

Rigorous training module, similar to the PFIEV program (Programme de Formation d'Ingénieurs d'Excellence du Vietnam).

\section{Training Content}

\begin{itemize}
	\item Theory of electrical / electronic circuits
	\item Measurement technique and smart sensors
	\item Industrial communication networks
	\item Programming technique for chips, microcontrollers, microprocessors
	\item Classical and modern control techniques
	\item Programming technique for industrial automation control systems
	\item Electrical drives and power electronics
	\item Robotics
	\item Neural networks and artificial intelligence
\end{itemize}

\section{Expected Outcomes}

\subsection{Knowledge}

\begin{itemize}
	\item Management and monitoring of technological projects
	\item Consultancy, design and development of automation projects
	\item Operation, maintenance of automatic production lines 
	\item Integration devices to set up control systems
	\item Design, manufacturing and accreditation of measurement and control devices
	\item Research and development of intelligent automatic devices and modern control systems toward industry 4.0, internet-of-thing and artificial intelligence
\end{itemize}

\subsection{Skills}

\begin{itemize}
	\item Presentation
	\item Teamwork
	\item Time management
	\item Entrepreneurship
\end{itemize}

\subsection{Foreign language}

\begin{itemize}
	\item Fluent of English for communication and profession
	\item Passing TOEIC exam with at least 500 point before graduation
\end{itemize}

\end{document} 

